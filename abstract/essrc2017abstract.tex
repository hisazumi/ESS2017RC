\documentclass[xelatex,a4paper,ja=standard,jafont=hiragino-pron]{bxjsarticle}

\newcommand*{\team}[4]{\hrulefill\\{\Large #1}\vspace{0.5em}\\{#2}\\{#3}\vspace{0.5em}\\{#4}\vspace{1em}\\}
\pagestyle{empty}
\parindent = 0pt

\begin{document}
{\huge \centering ESSロボットチャレンジ2017チームポスター概要}\\
%
\team{マイクロクアッドコプタ自律制御システムの開発}{Shin Quoda(九州大学大学院システム情報科学府情報知能工学専攻社会情報システム工学コース)}{中村 隼也,高木 奏}{我々はマイクロクアッドコプタの自律制御に取り組んだ.機体には加速度センサやジャイロセンサが搭載されているが,カメラモジュールや他のセンサを搭載することは困難となっている.制御の主な手法としては,PCと機体のBluetooth通信を用いた自動コントローラ制御を実装した.この中で,ステートマシンによる挙動の変化やスロットルのPID制御を行った.また自作のGUIを用い,センサ値の変化を可視化することで,PID制御の動作確認を行った.}    
%
\team{N/A}{おざわフレンズ\\(関東学院大学 †理工学部理工学科情報ネット・メディアコース,\\‡大学院工学研究科情報学専攻,§大学院工学研究科電気工学専攻)}{今井 祐介†, 大橋 勇馬†, 小澤 鉄平†, 杉山 憂†, 田村 裕人†, 藤間 魁人†, 前嶋 恵輔†, 松下 友城†,\\藤岡 幹‡, 曽我 健太§}{我々は、提示された課題の実現に向けて、PID制御を用いた方法について提案する。主にライントレースについては読み取ったラインの値、ARマーカーについてはマーカーの判別、迷路状のコースについては座標データと地図データの誤差をもとにPID制御を行う。自動運転についても同様に、事前に与えられた地図データを元に座標データとの誤差を判定し、PID制御を行って網羅できるコース取りを行えるように設定を行う。}
%
\team{MRチケット開発に基づいたローバー競技ロボット開発}{se$\alpha$sT(東海大学†情報通信学部組込みソフトウェア工学科, ‡大学院情報学研究科)}{海老原 秀亮‡,茂木 康太郎‡,赤松 雅征†,荻野 稜†,金子 健雄†,小林 駿†,紺野貴宏†,齋藤 隆弘†,\\田村 佳愛†,長谷川 愛美里†,長谷川 源†,森本 実沙†,渡辺 晴美†‡}{本報告ではMRチケットと呼ぶ開発方法をロボット開発に適用する.本開発方法は,ムービーチケット開発とリスクマーキットで構成される.前者は文字主体の課題からフレームワークに基づき内部要因と環境要因を抽出し,簡易的なムービーを作成する.これにより完成イメージをメンバー全員で共有でき,機能要件抽出を容易にする.ムービーを議論共有した後,チケットマップに整理しチケットとして発行する.後者はシステム完成度向上の障害である開発遅延防止のためにリスクマーキットを使用する. 発行されたチケットはタスクボードにより管理し,開発状況を可視化させる.本方法の適用により,開発者間の完成イメージ共有を目指す.}
%
\newpage\team{trelloによるスクラムの実践と無線通信を用いた\\ARマーカの読み取りと制御}{汐月研究室(東京電機大学ロボットメカトロニクス学科)}{渡辺吉城,番塲隆礼,田島朋輝,井上陸也,柴谷修司,汐月哲夫}{我々は,zumoの撮影画像からARマーカ情報を抽出し、これを使ってzumoを制御すること検討している。データ処理はパソコン側で行うため、今回はzumoからパソコンにの画像データを無線通信で転送するプログラムを実装した。zumoの制御プログラムはC言語,パソコン側のARマーカ判別プログラムはjavaと開発言語が異なるので、プログラム開発作業の円滑化・効率化のためにtrelloを用いたスクラムを活用した。}
%
\team{ESSロボットチャレンジ参加報告}{ふわっと(東京都市大学メディア情報学部情報システム学科)}{鈴木恵武, 菊池聡, 青戸涼, 大熊陽, 仲川 拓馬, 小倉信彦}{本報告では, チームふわっとで参加したESSロボットチャレンジのローバトライアル競技における, チームの取り組みについて紹介する. チーム開発過程の概要と, その過程で起こった課題や学習について整理し示す.}
\end{document}
